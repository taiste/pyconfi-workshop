\documentclass[13pt,xcolor=dvipsnames]{beamer}
\usepackage[utf8]{inputenc}
\usepackage{palatino}
\usepackage{listings}
\usecolortheme[named=Black]{structure}
\usefonttheme{serif}
\usetheme{Pittsburgh}

\title{Going Concurrent with Python}
\author{Mikko Harju}
\institute{Taiste}

\begin{document}

\begin{frame}
\titlepage
\end{frame}

\begin{frame}
    \frametitle{About me}
    \begin{itemize}
        \item I'm Mikko Harju, Technology Director at Taiste
        \item Python user since 2008
        \item Functional programming enthusiast
    \end{itemize}
\end{frame}

\begin{frame}
    \frametitle{Agenda}
    \begin{itemize}
        \item We'll look slightly at \emph{threading} and more at \emph{processes} and how do they differ when
            implementing concurrency in Python.
    \end{itemize}
\end{frame}

\begin{frame}
    \frametitle{Threads}
    \begin{itemize}
        \item Threads have classically been the basic construct for parallel programming.
        \item They live within the process boundary, hence sharing the memory space.
        \item This makes it fairly trivial to transfer information from one thread to another.
        \item However, hilarity ensues when mutable state enters the equation\ldots
    \end{itemize}
\end{frame}

\begin{frame}
    \frametitle{Threads}
    \begin{itemize}
        \item To prevent problems when writing data from multiple threads to a shared state variable we need to
            introduce different kinds of \emph{locking mechanisms}.
        \item Possibilities include e.g. mutexes, semaphores and critical sections.
        \item However, if the threads only access the data in read-only way, there is no need for locking.
        \item This is called ''Shared-nothing`` -approach. For example Erlang kinda has this with its ''lightweight
            processes``.
    \end{itemize}
\end{frame}

\begin{frame}
    \frametitle{GIL}
    \begin{itemize}
        \item GIL is a mechanism to ensure that threads co-operate nicely with non-thread safe constructs 
        \item This is done by locking down the interpreter by giving exclusive access to one thread at a time.
        \item This effectively means that only one CPU bound thread is actually doing anything useful in one
            python interpreter process at a time.
        \item When doing I/O, the interpreter can release the lock. The interpreter also has periodic checks to go along
            with this to make it possible to parallelize CPU bound threads.
    \end{itemize}
\end{frame}

\begin{frame}
    \frametitle{GIL}
    \begin{itemize}
        \item When we add more processors (or cores) to the mix, things get more complicated.
        \item N threads can be scheduled simultaneously on N processors, making them all compete over the GIL. Nice.
        \item So really, threads are not the right way to go in Python.
        \item There are some use cases where they might come in handy, where the problem is more I/O bound than CPU
            bound.
        \item But let us concentrate on the more fruitful (and the right way if you ask me) of doing real multiprocessor concurrency: \emph{processes}
    \end{itemize}
\end{frame}

\begin{frame}
    \frametitle{Processes}
    \begin{itemize}
        \item Processes are OS backed construct.
        \item Each process has its own memory space, stack, registers and that kind of stuff.
        \item Scheduling is performed by the operating system.
    \end{itemize}
\end{frame}

\begin{frame}
    \frametitle{Let's do this!}
    \begin{itemize}
        \item Nothing beats practice, so let's do parallel image processing with PIL using the multiprocessing package
            and see what we come up with. 
        \item Prepare your terminals :-)
    \end{itemize}
\end{frame}

\end{document}

